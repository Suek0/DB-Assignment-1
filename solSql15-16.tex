\documentclass{article}
\usepackage[margin=2cm]{geometry}
\usepackage[utf8]{inputenc}
\usepackage[spanish]{babel}
\usepackage{xcolor}
\usepackage{listings}

\lstset{
  basicstyle=\small,
  identifierstyle=\ttfamily,
  keywordstyle=\bfseries\ttfamily\color[rgb]{0,0,1},
  stringstyle=\ttfamily\color[rgb]{0.627,0.126,0.941},
  morekeywords={REFERENCES}
}

\title{Solución ejercicios de SQL}
\author{
  Iago Ossorio Astray \\
  \texttt{iagooa@protonmail.com}
}

\begin{document}
  \maketitle
  \begin{enumerate}
    \item Halla los empleados que tienen una comisión superior a la mitad de su
      salario.
      \lstinputlisting[language=SQL, frame=single]{queries/q1.sql}

    \item Halla los empleados que no tienen comisión, o que la tengan menor o
      igual que el 25\% de su salario.
      \lstinputlisting[language=SQL, frame=single]{queries/q2.sql}

    \item Obtén los empleados que no son supervisados por ningún otro.
      \lstinputlisting[language=SQL, frame=single]{queries/q3.sql}

    \item Para los empleados que tengan comisión, obtén sus nombres y el
      cociente entre su salario y su comisión (excepto cuando la comisión sea
      cero), ordenando el resultado por nombre.
      \lstinputlisting[language=SQL, frame=single]{queries/q4.sql}

    \item Para los empleados que tengan como jefe a un empleado con código mayor
      que el suyo, obtén los que reciben de salario más de 1000 y menos de 2000,
      o que están en el departamento 30.
      \lstinputlisting[language=SQL, frame=single]{queries/q5.sql}

    \item Obtén el nombre, salario, comisión y salario total (salario+comisión,
      si tiene comisión) de los empleados con salario total superior a 2300.
      \lstinputlisting[language=SQL, frame=single]{queries/q6.sql}

    \item Obtén los puestos de trabajo que hay en cada departamento, de forma
      que no se repitan filas.
      \lstinputlisting[language=SQL, frame=single]{queries/q7.sql}

    \item Obtén el salario más alto de la empresa, el total destinado a
      comisiones y el número de empleados.
      \lstinputlisting[language=SQL, frame=single]{queries/q8.sql}

    \item Halla el nombre el último empleado por orden alfabético.
      \lstinputlisting[language=SQL, frame=single]{queries/q9.sql}

    \item Halla el salario más alto, el más bajo, y la diferencia entre ellos.
      \lstinputlisting[language=SQL, frame=single]{queries/q10.sql}

    \item ¿Cúantos empleos diferentes, cuántos empleados, y cuántos salarios
      diferentes encontramos en el departamento 30, y a cuánto asciende la suma
      de salarios de dicho departamento?
      \lstinputlisting[language=SQL, frame=single]{queries/q11.sql}

    \item ¿Cúantos empleados tiene el departamento 20?
      \lstinputlisting[language=SQL, frame=single]{queries/q12.sql}

    \item ¿Cúantos empleados tienen comisión?
      \lstinputlisting[language=SQL, frame=single]{queries/q13.sql}

    \item ¿Qué empleos distintos encontramos en la empresa, y cúantos empleados
      desempeñan cada uno de ellos?
      \lstinputlisting[language=SQL, frame=single]{queries/q14.sql}

    \item Halla la suma de salarios de cada departamento, junto con el código
      del departamento.
      \lstinputlisting[language=SQL, frame=single]{queries/q15.sql}

    \item Para cada departamento muestra cuántos proyectos controla, junto con
      el código del departamento.
      \lstinputlisting[language=SQL, frame=single]{queries/q16.sql}

    \item Muestra los proyectos en los que trabaja al menos tres empleados y
      cuántas horas trabajan en dicho proyectos.
      \lstinputlisting[language=SQL, frame=single]{queries/q17.sql}

    \item Para cada departamento muestra cuántos proyectos controla en cada
      ciudad.
      \lstinputlisting[language=SQL, frame=single]{queries/q18.sql}

    \item Para cada departamento muestra cuántos empleados tiene que ganen más
      de 1500.
      \lstinputlisting[language=SQL, frame=single]{queries/q19.sql}

    \item Muestra los departamentos que tienen un salario mínimo mayor o igual
      a 1000. Muestra su código y cuántos empleados tiene.
      \lstinputlisting[language=SQL, frame=single]{queries/q20.sql}

    \item Muestra los departamentos y los trabajos donde hay por lo menos dos
      trabajadores con ese puesto de trabajo.
      \lstinputlisting[language=SQL, frame=single]{queries/q21.sql}

    \item Halla los datos de los empleados cuyo salario es mayor que el del
      empleado de código 7934, ordenando por el salario.
      \lstinputlisting[language=SQL, frame=single]{queries/q22.sql}

    \item Obtén los empleados que trabajan en Dallas o New York.
      \lstinputlisting[language=SQL, frame=single]{queries/q23.sql}

    \item Halla los empleados cuyo salario supera o coincide con la media del
      salario de la empresa.
      \lstinputlisting[language=SQL, frame=single]{queries/q24.sql}

    \item Obtén los empleados del departamento 10 que tienen el mismo empleo
      que alguien del departamento de Ventas. Desconocemos el código de dicho
      departamento.
      \lstinputlisting[language=SQL, frame=single]{queries/q25.sql}

    \item Halla los empleados que tienen por lo menos un empleado a su mando,
      ordenados inversamente por nombre.
      \lstinputlisting[language=SQL, frame=single]{queries/q26.sql}

    \item Halla los empleados que no tienen ningún empleado a su mando.
      \lstinputlisting[language=SQL, frame=single]{queries/q27.sql}

    \item Obtén todos los departamentos sin empleados.
      \lstinputlisting[language=SQL, frame=single]{queries/q28.sql}

    \item Muestra el código del empleado o empleados que más horas trabaja(n)
      en cada proyecto.
      \lstinputlisting[language=SQL, frame=single]{queries/q29.sql}

    \item Obtén los empleados cuyo salario supera al de sus compañeros de
      departamento. Si hay algún departamento donde dos, o más, empleados tienen
      el salario más alto, entonces nadie supera a sus compañeros.
      \lstinputlisting[language=SQL, frame=single]{queries/q30.sql}

    \item Para cada puesto de trabajo el/los empleados que más ganan.
      \lstinputlisting[language=SQL, frame=single]{queries/q31.sql}

    \item ¿Qué empleados trabajan en ciudades de más de cinco letras? Ordena el
      resultado inversamente por ciudades y normalmente por los nombres de los
      empleados.
      \lstinputlisting[language=SQL, frame=single]{queries/q32.sql}

    \item Para cada empleado muestra los proyectos en los que trabaja. Muestra
      el nombre del empleado y el nombre del proyecto.
      \lstinputlisting[language=SQL, frame=single]{queries/q33.sql}

    \item Muestra los proyectos controlados por cada departamento. Muestra el
      nombre del departamento y el nombre del proyecto. Deben aparecer todos los
      departamentos, incluso si no controla ningún departamento.
      \lstinputlisting[language=SQL, frame=single]{queries/q34.sql}

    \item Obtén un listado en el que se reflejen los empleados y los nombres de
      sus jefes. En el listado deben aparecer todos los empleados, aunque no
      tengan jefe.
      \lstinputlisting[language=SQL, frame=single]{queries/q35.sql}

    \item Los nombres de empleados contratados antes que su jefe.
      \lstinputlisting[language=SQL, frame=single]{queries/q36.sql}

    \item Para cada departamento, muestra los empleados que trabajan en
      proyectos controlados por él. Muestra el nombre del departamento y el
      código de los empleados.
      \lstinputlisting[language=SQL, frame=single]{queries/q37.sql}

    \item Obtén el código de empleado, el nombre, el salario, el código del
      proyecto y las horas que le dedica cada empleado vinculado a algún
      proyecto, ordenado por el código del empleado.
      \lstinputlisting[language=SQL, frame=single]{queries/q38.sql}

    \item ¿Cúantos empleados hay en cada departamento, y cuál es la media del
      salario de cada uno? Indique el nombre del departamento para clarificar el
      resultado.
      \lstinputlisting[language=SQL, frame=single]{queries/q39.sql}

    \item Muestra la suma de salarios de los empleados de cada departamento que
      tienen un salario superior al salario medio de la empresa. Muestra el
      nombre del departamento.
      \lstinputlisting[language=SQL, frame=single]{queries/q40.sql}

    \item Para cada proyecto controlado por el departamento 30, indica su número
      , ciudad, número de empleados participantes en el proyecto, las horas
      dedicadas por el empleado que más ha trabajado, y las horas dedicadas por
      el que menos ha trabajado, y la diferencia entre ellas.
      \lstinputlisting[language=SQL, frame=single]{queries/q41.sql}

    \item Por cada departamento muestra total de horas trabajadas en proyectos
      por los empleados de cada puesto de trabajo. Muestra el código de
      departamento y el nombre del puesto de trabajo
      \lstinputlisting[language=SQL, frame=single]{queries/q42.sql}

    \item Para cada empleado muestra su nombre y cuántos empleados supervisa de
      cada puesto de trabajo.
      \lstinputlisting[language=SQL, frame=single]{queries/q43.sql}

    \item Muestra para cada proyecto cuántas horas trabajan en total todos los
      empleados que tienen un salario superior al salario medio de la empresa.
      Muestra el nombre del proyecto.
      \lstinputlisting[language=SQL, frame=single]{queries/q44.sql}

    \item Para cada departamento, muestra su código, su nombre, el salario más
      alto y más bajo que cobran sus empleados, la diferencia entre estos dos
      salarios, y el número de proyectos a cargo del departamento.
      \lstinputlisting[language=SQL, frame=single]{queries/q45.sql}

    \item Considerando empleados con salario menor de 5000, halla la media de
      los salarios de los departamentos cuyo salario mínimo supera a 900.
      Muestra también el código y el nombre de los departamentos.
      \lstinputlisting[language=SQL, frame=single]{queries/q46.sql}

    \item Lista los empleados que tengan el mayor salario de su departamento,
      mostrando el nombre del empleado, su salario y el nombre del departamento.
      \lstinputlisting[language=SQL, frame=single]{queries/q47.sql}

    \item El puesto de trabajo con el salario medio más alto.
      \lstinputlisting[language=SQL, frame=single]{queries/q48.sql}

    \item Para cada supervisor, muestra su subordinado(s) que más gana.
      \lstinputlisting[language=SQL, frame=single]{queries/q49.sql}

    \item Deseamos saber cuántos empleados supervisa cada jefe. Para ello, obtén
      un listado en el que se reflejen el código y el nombre de cada jefe, junto
      al número de empleados que supervisa directamente. Como puede haber
      empleados sin jefe, para estos se indicará sólo el número de ellos, y los
      valores restantes (código y nombre del jefe) se dejarán como nulos.
      \lstinputlisting[language=SQL, frame=single]{queries/q50.sql}

    \item Hallar el/los departamento(s) cuya suma de salarios sea la más alta,
      mostrando esta suma de salarios y el nombre del departamento
      \lstinputlisting[language=SQL, frame=single]{queries/q51.sql}

    \item Obtén los datos de los empleados que cobren los dos mayores salarios
      de la empresa. (Nota: Procure hacer la consulta de forma que sea fácil
      obtener los empleados de los N mayores salarios).
      \lstinputlisting[language=SQL, frame=single]{queries/q52.sql}

    \item Obtén las localidades que no tienen departamentos sin empleados y en
      las que trabajen al menos cuatro empleados. Indica también el número de
      empleados que trabajan en esas localidades. (Nota: Por ejemplo, puede que
      en A Coruña existan dos departamentos, uno con más de cuatro empleados y
      otro sin empleados, en tal caso, A Coruña no debe aparecer en el resultado
      , puesto que tiene un departamento SIN EMPLEADOS, a pesar de tener otro
      con empleados Y tener más de cuatro empleados EN TOTAL. ATENCIÓN, la
      restricción de que tienen que ser cuatro empleados se refiere a la
      totalidad de los departamentos de la localidad).
      \lstinputlisting[language=SQL, frame=single]{queries/q53.sql}

    \item Obtén un listado de todos los empleados (código, nombre) donde
      aparezca el total de horas dedicado a proyectos. Deben aparecer todos los
      empleados aunque no trabajen en proyectos.
      \lstinputlisting[language=SQL, frame=single]{queries/q54.sql}

    \item Nombre del departamento(s) que tiene(n) el mayor número de
      supervisores.
      \lstinputlisting[language=SQL, frame=single]{queries/q55.sql}

    \item Nombres de empleados que trabajan solos en algún proyecto.
      \lstinputlisting[language=SQL, frame=single]{queries/q56.sql}

    \item Para cada empleado, número, nombre y contar cuantos ganan menos que él
      (si no hay ninguno, debe aparecer un 0).
      \lstinputlisting[language=SQL, frame=single]{queries/q57.sql}

    \item Para cada empleado, número, nombre y contar cuantos (descontando a él
      mismo) ganan lo mismo o menos que él (si no hay ninguno, debe aparecer un 0).
      \lstinputlisting[language=SQL, frame=single]{queries/q58.sql}

    \item Para cada jefe mostrar cuántos empleados supervisa en cada
      departamento (los empleados supervisados no tienen porque ser del mismo
      departamento que el supervisor). Mostrar nombre de empleado y código
      departamento (de los empleados supervisados).
      \lstinputlisting[language=SQL, frame=single]{queries/q59.sql}

    \item Idem mostrando todos los empleados, y en aquellos que no son jefes,
      mostrando un cero en el número de empleados supervisados.
      \lstinputlisting[language=SQL, frame=single]{queries/q60.sql}

    \item Para cada departamento que tenga, por lo menos, dos empleados sin
      comisión, muestra el nombre del departamento, y cuántos empleados tiene en
      total (con y sin comisión).
      \lstinputlisting[language=SQL, frame=single]{queries/q61.sql}

    \item ¿Cúal es la ciudad (o ciudades, si hay más de una) en la que hasta el
      momento se han trabajado más horas en proyectos? Indica el nombre de la
      ciudad, y el número de horas.
      \lstinputlisting[language=SQL, frame=single]{queries/q62.sql}

    \item Muestra para cada proyecto su código y, de los empleados que trabajan
      en dicho proyecto, el nombre del empleado que más gana de cada puesto de
      trabajo.
      \lstinputlisting[language=SQL, frame=single]{queries/q63.sql}

    \item Para cada empleado, obtén su código y su nombre, y el código y nombre
      del proyecto al que dedica más horas ese empleado. Muestra también las
      horas, y ordena el resultado por nombre de empleado.
      \lstinputlisting[language=SQL, frame=single]{queries/q64.sql}

    \item Considerando sólo los empleados que tienen el máximo salario de cada
      puesto de trabajo, muestra para cada departamento su nombre y cuántos de
      dichos empleados trabajan en ese departamento.
      \lstinputlisting[language=SQL, frame=single]{queries/q65.sql}

    \item Para cada empleado muestra su nombre, cuántos empleados supervisa en
      total, cuántos con comisión y cuántos sin comisión. Si el empleado no
      supervisa a nadie, o no supervisa empleados sin/con comisión, se debe
      mostrar un cero en el número correspondiente.
      \lstinputlisting[language=SQL, frame=single]{queries/q66.sql}

    \item Muestra el nombre y trabajo de los empleados que son supervisores y
      que tienen el mismo trabajo que todos sus subordinados.
      \lstinputlisting[language=SQL, frame=single]{queries/q67.sql}

    \item Para cada departamento muestra cuántos jefes tiene y cuántos
      subordinados tienen esos jefes (los subordinados no tienen porque ser del
      mismo departamento que el jefe). Deben aparecer todos los departamentos.
      \lstinputlisting[language=SQL, frame=single]{queries/q68.sql}

    \item Muestra el nombre de los departamentos con más de tres empleados de
      los cuales al menos dos son jefes.
      \lstinputlisting[language=SQL, frame=single]{queries/q69.sql}

    \item Para cada jefe, muestra su nombre y cuántos empleados supervisa en
      departamentos diferentes al suyo. Si un jefe no supervisa a ningún
      empleado de otro departamento, muestra un cero.
      \lstinputlisting[language=SQL, frame=single]{queries/q70.sql}
  \end{enumerate}
\end{document}
